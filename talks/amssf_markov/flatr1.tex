%%% Flat Reidemiester 1 moves, with arcs labeled as in original paper
\documentclass{standalone}

\usepackage{amsmath}
\input{markov_header.tex}

\usepackage{tikz}
\usetikzlibrary{calc,decorations,arrows.meta,decorations.markings,decorations.pathmorphing,backgrounds,hobby,knots}

\input{markov_tikz.tex}

\begin{document}
\begin{tikzpicture}
  \begin{scope}[xshift=-1.25in]
    \small
    \node (x) at (0,0) {};
    \node (e1) at (-1,-1) {};
    \node (e2) at (1,-1) {};
    \node (e3) at (0,1) {};

    \draw[very thick,startarc={-1}{\arccolor}] (e1.center) .. controls +(60:.5) and +(180:.5) .. (x.center)
    node [right,pos=0.2] {\contour{white}{\(a_2\)}};
    \draw[very thick,endarc={1}{\rootcolor}]
    (x.center) .. controls +(0:.5) and +(120:.5) .. (e2.center)
    node [right,pos=0.8] {\contour{white}{\(a_1\)}};
  \end{scope}

  \begin{scope}[xshift=1.25in]
    \small
    \node (x) at (0,0) {};
    \node (e1) at (-1,-1) {};
    \node (e2) at (1,-1) {};
    \node (e3) at (0,1) {};

    \draw[very thick, fwdarcs={-1}{0.5}{\arccolor}{\arccolor}]
    (e1.center) -- (x.center)
    node [left,pos=0.8] {\contour{white}{\(a_3\)}}
    node [right,pos=0.1] {\contour{white}{\(a_2\)}};

    \draw[very thick, fwdarcs={-1}{0.5}{\arccolor}{\arccolor}]
    (x.center) -- (e2.center)
    node [left,pos=0.5] {\contour{white}{\(a_4\)}}
    node [right,pos=0.7] {\contour{white}{\(a_1\)}};
    
    \draw[very thick,startarc={-1}{\arccolor}] (x.center) .. controls (.5,.5) and (.5,1) .. (e3.center)
    node [right,pos=0.4] {\contour{white}{\(a_5\)}};
    \draw[very thick,endarc={1}{\rootcolor}] (e3.center) .. controls (-.5,1) and (-.5,.5) .. (x.center)
    node [right,pos=0.6] {\contour{white}{\(a_6\)}};

    \draw[fill=black] (x) circle [radius=3pt];
  \end{scope}

  \draw[->] (-.6in,0.1) -- (.6in,0.1)
  node [above,pos=0.5] {\(\RIp(M,a_1)\)};
  \draw[->] (.6in,-0.1) -- (-.6in,-0.1)
  node [below,pos=0.5] {\(\RIm(M',a_6)\)};
  
\end{tikzpicture}
\end{document}
