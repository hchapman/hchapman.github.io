% LaTeX Curriculum Vitae Template
%
% Copyright (C) 2004-2009 Jason Blevins <jrblevin@sdf.lonestar.org>
% http://jblevins.org/projects/cv-template/
%
% You may use use this document as a template to create your own CV
% and you may redistribute the source code freely. No attribution is
% required in any resulting documents. I do ask that you please leave
% this notice and the above URL in the source code if you choose to
% redistribute this file.

\documentclass[letterpaper]{article}

\usepackage{hyperref}
\usepackage{geometry}
\usepackage{amsmath}
\usepackage{amsfonts}

% Comment the following lines to use the default Computer Modern font
% instead of the Palatino font provided by the mathpazo package.
% Remove the 'osf' bit if you don't like the old style figures.
%\usepackage[T1]{fontenc}
%\usepackage[sc,osf]{mathpazo}

% Set your name here
\def\name{Harrison Chapman}

% Replace this with a link to your CV if you like, or set it empty
% (as in \def\footerlink{}) to remove the link in the footer:
\def\footerlink{http://hchapman.github.io/static/CV.pdf}

% The following metadata will show up in the PDF properties
\hypersetup{
  colorlinks = true,
  urlcolor = black,
  pdfauthor = {\name},
  pdfkeywords = {economics, statistics, mathematics},
  pdftitle = {\name: Curriculum Vitae},
  pdfsubject = {Curriculum Vitae},
  pdfpagemode = UseNone
}

\geometry{
  body={6.5in, 8.5in},
  left=1.0in,
  top=1.25in
}

% Customize page headers
\pagestyle{myheadings}
\markright{\name}
\thispagestyle{empty}

% Custom section fonts
\usepackage{sectsty}
\sectionfont{\rmfamily\mdseries\Large}
\subsectionfont{\rmfamily\mdseries\itshape\large}

% Other possible font commands include:
% \ttfamily for teletype,
% \sffamily for sans serif,
% \bfseries for bold,
% \scshape for small caps,
% \normalsize, \large, \Large, \LARGE sizes.

% Don't indent paragraphs.
\setlength\parindent{0em}

% Make lists without bullets
\renewenvironment{itemize}{
  \begin{list}{}{
    \setlength{\leftmargin}{1.5em}
  }
}{
  \end{list}
}

\begin{document}

% Place name at left
{\huge \name}\\
{\large\sc Curriculum Vitae}

% Alternatively, print name centered and bold:
%\centerline{\huge \bf \name}

\vspace{0.25in}

\begin{minipage}{0.45\linewidth}
  \href{http://www.uga.edu/}{University of Georgia} \\
  Department of Mathematics \\
  Boyd Graduate Studies Research Center \\
  Athens, GA 30602
\end{minipage}
\begin{minipage}{0.45\linewidth}
  \begin{tabular}{ll}
    Phone: & (706) 542-2619 \\
    Email: & \href{mailto:hchapman@uga.edu}{\tt hchapman@uga.edu} \\
    Homepage: & \href{http://hchapman.github.io/}{\tt http://hchapman.github.io/} \\
  \end{tabular}
\end{minipage}

\section*{Education}

\begin{itemize}
  \item B.A.\ Mathematics and Computer Science, Bowdoin College,
    2011.\\
    - \quad Cum laude \\
    - \quad Honors in Mathematics (advisor: Thomas Pietraho)
  \item M.S.\ Mathematics, University of Georgia, 2015.
  \item Ph.D.\ Mathematics, University of Georgia, 5th year (ABD), expected May
    2017.\\
   - \quad Advisor: Jason Cantarella.
\end{itemize}

\section*{Teaching}

\begin{enumerate}
\item Instructor, Calculus for Science and Engineering I (MATH2250), University of Georgia (Spring 2014, Spring 2016)
\item Instructor, Precalculus (MATH1113), University of Georgia (Fall 2013, Fall 2015)
\item Writing Intensive Program (WIP) teaching assistant for a lab and robotics-focused Calculus I (MATH2250) (Fall 2015)
\item Graduate assistant, Online
  Precalculus (MATH1113E), University System of Georgia. (Fall 2014--Spring 2015)
\item Recitation instructor, Analytic Geometry and Calculus
  (MATH2200), University of Georgia (Fall 2011, Spring 2012, Fall 2014)
\end{enumerate}

\section*{Publications}

\begin{enumerate}
\item Asymptotics of knot and link diagrams. In preparation.\\
  \textit{I prove 0-1 laws for tangles in diagrams using techniques
    and results from the study of combinatorial maps.}
\item On probabilistic unknotting algorithms using only local information.\\
  with J.\ Cantarella, E.\ Lybrand. In preparation.\\
  \textit{There is a simple deterministic algorithm for converting any
    knot diagram to a diagram of the unknot by crossing changes, but
    it requires global information about the diagram. Using our
    enumeration of diagrams, we consider whether there is a
    probabilistic algorithm for unknotting a diagram as such with only
    local information. This question is relevant to the untangling of
    DNA by topoisomerase action.}
\item Knot Probabilities in Random Diagrams.\\
  with J.\ Cantarella, M.\ Mastin. \href{http://arxiv.org/abs/1512.05749}{arXiv:1512.05749}.\\
  \textit{We compute the exact probability that a random diagram of
    $n$ crossings has a given knot type (for $n \leq 10$) by directly
    enumerating and classifying the knot diagrams of $n$
    crossings. Our enumeration method is based on identifying knot
    diagrams with a class of embedded 4-regular planar graphs.}
\item A Group-theoretic Approach to Human Solving Strategies in
  Sudoku.\\
  with M.\ Rupert (Mentor: E.\ Arnold). \textit{Colonial Academic
    Alliance
    Undergraduate Research Journal} (2012) vol 3, article 3.\\
  \textit{Paper produced during an NSF REU at James Madison University
    in 2010. We quantify the data of Sudoku board states by
    considering which numbers} cannot \textit{go in a given cell and
    consider how a typical player's solving strategies are a group
    acting on this set of states.}
\item On orbital varieties of type A. \\
  Advisor: T.\ Pietraho. \textit{Honors thesis}, (2011). Bowdoin
  College.\\
  \textit{Thesis on the Smith conjecture on orbital varieties of
    nilpotent orbits in the Lie group $GL_n$.  Outlines the
    correspondence between Young tableaux and orbital varieties and
    concludes with conditions for which certain shapes of Young
    tableaux will admit orbital varities which are not Smith.}
\end{enumerate}

\section*{Software}

\begin{enumerate}
\item \texttt{plCurve}. C and Python. Piecewise-linear curve and link diagram library.\\
  with T.\ Ashton, J.\ Cantarella, M.\ Mastin.
  \textit{My primary contribution has been a Python interface to the C
  library code.}
\item Reverb. Java and C. An Android app which uses PulseAudio to
  control volume and audio streams on Linux computers. \\
  \url{https://github.com/hchapman/reverb}
\end{enumerate}

\section*{Talks}

\begin{enumerate}
\item Joint Mathematics Meetings, January 2016.\\
  MAA Session on Mathematical Modeling in the Undergraduate Classroom. \\
  \textit{A robotics-based calculus class}.
\item Joint Mathematics Meetings, January 2016.\\
  AMS Session on General Topics. \\
  \textit{Asymptotic laws for knot diagrams}.
\item Tulane University, Geometry Seminar, October 2015.\\
  \textit{Asmyptotic laws for knot diagrams}.
\item AMS Fall Western Sectional Meetings, October 2015.\\
  Special Session on Algebraic and Combinatorial Structures in Knot Theory. \\
  \textit{Asymptotics of random knot diagrams}.
\item University of Georgia, Geometry Seminar, October 2015.\\
  \textit{The quantum harmonic oscillator}.
\item AMS Fall Southeastern Sectional Meetings, October 2015.\\
  Special Session on Topological Combinatorics. \\
  \textit{Asymptotics of random knot diagrams}.
\item University of British Columbia, Discrete Math Seminar, September 2015.\\
  \textit{Asmyptotic laws for knot diagrams}.
\item Simon Fraser University, Discrete Math Seminar, September 2015.\\
  \textit{Asmyptotic laws for knot diagrams}.
\item University of Georgia, Geometry Seminar, September 2015.\\
  \textit{Asmyptotic laws for knot diagrams}.
\item University of Georgia, Graduate Student Seminar, September 2015.\\
  \textit{How to count (a quick glance at analytic combinatorics)}.
\item University of Georgia, Mock AMS Conference, July 2015.\\
  \textit{Asymptotics of knot and link diagrams}.
\item AMS Spring Western Sectional Meetings, April 2015.\\
  Special Session on Inverse Problems and Related Mathematical Methods in Physics. \\
  \textit{Random knot diagrams}.
\item University of Georgia, Graduate Student Seminar, February
  2015.\\
  \textit{Virtual Knot Theory}.
\item University of Georgia, Geometry Seminar, January 2014.\\
  \textit{Random Planar Diagrams}.
\item University of Georgia, Graduate Student Topology Seminar,
  November 2014.\\
  \textit{The Poincar\'e homolgy sphere as the link of a singularity}.
\item University of Georgia, Research Group on
  Minimal Surfaces, November 2014.\\
  \textit{Discrete Ricci Flow}.
\item University of Georgia, Mock AMS Conference, June 2014.\\
  \textit{The Tropical Grassmannian}.
\item University of Georgia, Mock AMS Conference, June 2013.\\
  \textit{Hope for slackers: Playing games to prove theorems}.
\item University of Georgia, Mock AMS Conference, June 2012.\\
  \textit{The Classification of Surfaces}.
\item University of Georgia, VIGRE Research Group on Minkowski's
  Geometry of Numbers, March 2012.\\
  \textit{Vinogradov's generalization of a theorem of Aubry-Thue}.
\item Joint Mathematics Meetings, January 2011.\\
  AMS-MAA-SIAM Special Session on Research in Mathematics by
  Undergraduates and Students in Post-Baccalaureate Programs.\\
  \textit{Packets, Solving Symmetries, and Sudoku}.
\item The Ohio State University, Young Mathematicians Conference,
  August 2010.\\
  \textit{Packets, Solving Symmetries, and Sudoku}.
\end{enumerate}

\bigskip

% Footer
\begin{center}
  \begin{footnotesize}
    Last updated: \today \\
    \href{\footerlink}{\texttt{\footerlink}}
  \end{footnotesize}
\end{center}

\end{document}

%%% Local Variables:
%%% mode: latex
%%% TeX-master: t
%%% End:
