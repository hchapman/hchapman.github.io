% LaTeX Curriculum Vitae Template
%
% Copyright (C) 2004-2009 Jason Blevins <jrblevin@sdf.lonestar.org>
% http://jblevins.org/projects/cv-template/
%
% You may use use this document as a template to create your own CV
% and you may redistribute the source code freely. No attribution is
% required in any resulting documents. I do ask that you please leave
% this notice and the above URL in the source code if you choose to
% redistribute this file.

\documentclass[letterpaper]{article}

\usepackage{hyperref}
\usepackage{geometry}
\usepackage{amsmath}
\usepackage{amsfonts}

% Comment the following lines to use the default Computer Modern font
% instead of the Palatino font provided by the mathpazo package.
% Remove the 'osf' bit if you don't like the old style figures.
%\usepackage[T1]{fontenc}
%\usepackage[sc,osf]{mathpazo}

% Set your name here
\def\name{Harrison Chapman}

% Replace this with a link to your CV if you like, or set it empty
% (as in \def\footerlink{}) to remove the link in the footer:
\def\footerlink{http://hchapman.github.io/static/CV.pdf}

% The following metadata will show up in the PDF properties
\hypersetup{
  colorlinks = true,
  urlcolor = black,
  pdfauthor = {\name},
  pdfkeywords = {mathematics, knot theory, random topology, statistical mechanics, physics},
  pdftitle = {\name: Curriculum Vitae},
  pdfsubject = {Curriculum Vitae},
  pdfpagemode = UseNone
}

\geometry{
  body={6.5in, 8.5in},
  left=1.0in,
  top=1.25in
}

% Customize page headers
\pagestyle{myheadings}
\markright{\name}
\thispagestyle{empty}

% Custom section fonts
\usepackage{sectsty}
\sectionfont{\rmfamily\mdseries\Large}
\subsectionfont{\rmfamily\mdseries\itshape\large}

% Other possible font commands include:
% \ttfamily for teletype,
% \sffamily for sans serif,
% \bfseries for bold,
% \scshape for small caps,
% \normalsize, \large, \Large, \LARGE sizes.

% Don't indent paragraphs.
\setlength\parindent{0em}

% Make lists without bullets
\renewenvironment{itemize}{
  \begin{list}{}{
    \setlength{\leftmargin}{1.5em}
  }
}{
  \end{list}
}

\begin{document}

% Place name at left
{\huge \name}\\
{\large\sc Curriculum Vit\ae}

% Alternatively, print name centered and bold:
%\centerline{\huge \bf \name}

\vspace{0.25in}

\begin{minipage}{0.45\linewidth}
  \href{http://www.uga.edu/}{University of Georgia} \\
  Department of Mathematics \\
  Boyd Graduate Studies Research Center \\
  Athens, GA 30602
\end{minipage}
\begin{minipage}{0.45\linewidth}
  \begin{tabular}{ll}
    Phone: & (706) 542-2619 \\
    Email: & \href{mailto:hchapman@uga.edu}{\tt hchapman@uga.edu} \\
    Homepage: & \href{http://hchapman.github.io/}{\tt http://hchapman.github.io/} \\
  \end{tabular}
\end{minipage}

\section*{Education}

\begin{itemize}
\item Ph.D.\ Mathematics, University of Georgia, 6th year (ABD), expected May
  2017.\\
  \textbullet \quad Advisor: Jason Cantarella.
\item M.S.\ Mathematics, University of Georgia, 2015.
\item B.A.\ Mathematics and Computer Science, Bowdoin College,
  2011.\\
  \textbullet \quad Cum laude \\
  \textbullet \quad Honors in Mathematics (advisor: Thomas Pietraho)
\end{itemize}

 \section*{Awards}

 \begin{itemize}
 \item B.J.\ Ball Scholarship, 2016
 \item UGA VIGRE Research Fellowship, 2012---2013, Summer 2014
 \item Bowdoin Faculty Scholar, 2007
 \end{itemize}

\section*{Teaching}

\subsection*{University of Georgia}

\begin{enumerate}
\item Instructor, Calculus for Science and Engineering I (MATH2250). \\
  Spring 2014, Spring 2016, Fall 2016)
\item Instructor, Precalculus (MATH1113). \\
  Fall 2013, Fall 2015
\item Graduate assistant for Topology Qualifying Exam problem session
  (volunteer). \\
  Summer 2016
\item Writing Intensive Program (WIP) teaching assistant for a lab and
  robotics-focused Calculus I (MATH2250). \\
  Fall 2015
\item Graduate assistant, Online Precalculus (MATH1113E). \\
  \emph{This was a course for all of the University System of Georgia}.\\
  Fall 2014--Spring 2015
\item Recitation instructor, Analytic Geometry and Calculus
  (MATH2200).\\
  Fall 2011, Spring 2012, Fall 2014
\item Grading:
  \begin{itemize}
  \item Introduction to Higher Mathematics (MATH 3200): Fall 2011, Fall 2013
  \item Graph Theory (MATH/CS 4690/6690): Spring 2012
  \item Modern Algebra and Geometry (MATH 4000): Spring 2012
  \end{itemize}

\end{enumerate}

\section*{Publications}

\begin{enumerate}
\item Markov chain sampling of random knot diagrams. In preparation. \\
  with A.\ Rechnitzer. \\
  \textit{We demonstrate a Markov Chain Monte Carlo sampler on the space of knot
    shadows and prove that it is ergodic. This provides a new way to uniformly
    sample knot diagrams and computationally enumerate the class of knot
    diagrams.}
\item Slipknotting in knot diagrams. In preparation. \\
  \textit{I apply pattern theorem results to knot diagrams to show that
    slipknots of any topological type appear almost surely in different classes
    of knot diagrams. Namely, I show that diagrams of unknots contain slipknots,
    a result which remains a conjecture for other models of knotting.}
\item Asymptotic laws for random knot diagrams. \emph{Submitted} 2016. \\
  Preprint: \href{http://arxiv.org/abs/1608.02638}{\tt arXiv:1608.02638}. \\
  \textit{I show a pattern theorem result for knot diagrams and use it to both
    prove the Frisch-Wasserman-Delbr\"uck conjecture for the model of knot
    diagrams and show that symmetries of knot diagrams are statistically
    insignificant.}
\item Knot probabilities in random diagrams.\\
  with J.\ Cantarella, M.\ Mastin. \\
  J. Phys. A: Math. Theor. 49 (2016), no. 40, p. 405001.\\
  DOI: \href{http://dx.doi.org/10.1088/1751-8113/49/40/405001}{\tt 10.1088/1751-8113/49/40/405001}.\\
  Preprint: \href{http://arxiv.org/abs/1512.05749}{\tt arXiv:1512.05749}. \\
  \textit{We compute the exact probability that a random diagram of
    $n$ crossings has a given knot type (for $n \leq 10$) by directly
    enumerating and classifying the knot diagrams of $n$
    crossings. Our enumeration method is based on identifying knot
    diagrams with a class of embedded 4-regular planar graphs.}
\item A Group-theoretic Approach to Human Solving Strategies in
  Sudoku.\\
  with M.\ Rupert (Mentor: E.\ Arnold). \\
  Colonial Academic Alliance Undergraduate Research Journal (2012) vol
  3, article 3.\\
  \textit{Paper produced during an NSF REU at James Madison University
    in 2010. We quantify the data of Sudoku board states by
    considering which numbers \emph{cannot} go in a given cell and
    consider how a typical player's solving strategies are a group
    acting on this set of states.}
\item On orbital varieties of type A. \\
  Advisor: T.\ Pietraho. Honors thesis (2011). Bowdoin
  College. \\
  \textit{Thesis on the Smith conjecture on orbital varieties of
    nilpotent orbits in the Lie group $GL_n$.  Outlines the
    correspondence between Young tableaux and orbital varieties and
    concludes with conditions for which certain shapes of Young
    tableaux will admit orbital varities which are not Smith.}
\end{enumerate}

\section*{Software}

\begin{enumerate}
\item \emph{LiveFit}. C++. Augmented reality projectile-tracking demonstration
  for use in calculus classes. \\
  \url{https://github.com/hchapman/LiveFit}
\item \texttt{plCurve}. C and Python. Piecewise-linear curve and link diagram library.\\
  with T.\ Ashton, J.\ Cantarella, M.\ Mastin.
  \textit{My primary contribution has been a Python interface to the C
    library code.} \\
  \url{http://www.jasoncantarella.com/wordpress/software/plcurve/}
\item \emph{Reverb}. Java and C. An Android app which uses PulseAudio to
  control volume and audio streams on Linux computers. \\
  \url{https://github.com/hchapman/reverb}
\end{enumerate}

\section*{External Talks}

\begin{enumerate}
\item Joint Mathematics Meetings 2017, \\
  MAA Invited Paper Session on Random Polygons and Knots. \\
  \textit{Slipknotting in the Knot Diagram Model}. Atlanta GA, January 2017.

\item Piedmont College, Annual Math and Physics Lecture, \\
  \textit{Random knots in physics and biology}. Demorest GA, November 2016.

\item 28\textsuperscript{th} International Conference on Formal Power Series and
  Algebraic Combinatorics, \\
  \textit{Asymptotic laws for knot diagrams}. Vancouver BC, July 2016.

\item IU Bloomington, Graduate Student Topology and Geometry Conference 2016, \\
  \textit{Asymptotic laws for knot diagrams}. Bloomington IN, April 2016.
  
\item Joint Mathematics Meetings 2016, \\
  MAA Session on Mathematical Modeling in the Undergraduate Classroom. \\
  \textit{A robotics-based calculus class}. Seattle WA, January 2016.

\item Joint Mathematics Meetings 2016, \\
  AMS Session on General Topics. \\
  \textit{Asymptotic laws for knot diagrams}. Seattle WA, January 2016.

\item Tulane University, Geometry Seminar, \\
  \textit{Asmyptotic laws for knot diagrams}. New Orleans LA, October 2015.

\item CSU Fullerton, AMS Fall Western Sectional Meetings 2015,\\
  Special Session on Algebraic and Combinatorial Structures in Knot Theory. \\
  \textit{Asymptotics of random knot diagrams}. Fullerton CA, October 2015,

\item University of Memphis, AMS Fall Southeastern Sectional Meetings 2015,\\
  Special Session on Topological Combinatorics. \\
  \textit{Asymptotics of random knot diagrams}. Memphis TN, October 2015.

\item University of British Columbia, Discrete Math Seminar, \\
  \textit{Asmyptotic laws for knot diagrams}. Vancouver BC, September 2015.
  
\item Simon Fraser University, Discrete Math Seminar, \\
  \textit{Asmyptotic laws for knot diagrams}. Burnaby BC, September 2015.

\item University of Saskatchewan, PIMS-USASK Graduate Summer School on Applied Combinatorics, \\
  \textit{Knot diagrams and blossom trees}. Saskatoon SK, May 2015.
  
\item University of Nevada, AMS Spring Western Sectional Meetings,\\
  Special Session on Inverse Problems and Related Mathematical Methods in Physics. \\
  \textit{Random knot diagrams}. Las Vegas NV, April 2015.
  
\item Joint Mathematics Meetings, \\
  AMS-MAA-SIAM Special Session on Research in Mathematics by
  Undergraduates and Students in Post-Baccalaureate Programs.\\
  \textit{Packets, Solving Symmetries, and Sudoku}. New Orleans LA, January 2011.
  
\item The Ohio State University, Young Mathematicians Conference, \\
  \textit{Packets, Solving Symmetries, and Sudoku}. Columbus OH, August 2010.
\end{enumerate}

\section*{Internal Talks}

\begin{enumerate}
  
\item University of Georgia, Geometry Seminar, \\
  \textit{Patterns in knot diagrams}. Athens GA, August 2016. 

\item University of Georgia, Geometry Seminar,\\
  \textit{The quantum harmonic oscillator}. Athens GA, October 2015.
  
\item University of Georgia, Geometry Seminar, \\
  \textit{Asmyptotic laws for knot diagrams}. Athens GA, September 2015.
  
\item University of Georgia, Graduate Student Seminar, \\
  \textit{How to count (a quick glance at analytic combinatorics)}. Athens GA, September 2015.
  
\item University of Georgia, Mock AMS Conference, \\
  \textit{Asymptotics of knot and link diagrams}. Athens GA, July 2015.

\item University of Georgia, Graduate Student Seminar,\\
  \textit{Virtual Knot Theory}. Athens GA, February 2015.
  
\item University of Georgia, Geometry Seminar,\\
  \textit{Random Planar Diagrams}. Athens GA, January 2014.
  
\item University of Georgia, Graduate Student Topology Seminar,\\
  \textit{The Poincar\'e homolgy sphere as the link of a singularity}. Athens GA, November 2014.
  
\item University of Georgia, Research Group on Minimal Surfaces, \\
  \textit{Discrete Ricci Flow}. Athens GA, November 2014.
  
\item University of Georgia, Mock AMS Conference, \\
  \textit{The Tropical Grassmannian}. Athens GA, June 2014.
  
\item University of Georgia, Mock AMS Conference, \\
  \textit{Hope for slackers: Playing games to prove theorems}. Athens GA, June 2013.
  
\item University of Georgia, Mock AMS Conference, \\
  \textit{The Classification of Surfaces}. Athens GA, June 2012.
  
\item University of Georgia, VIGRE Research Group on Minkowski's Geometry of Numbers, \\
  \textit{Vinogradov's generalization of a theorem of Aubry-Thue}. Athens GA, March 2012.

\end{enumerate}


\section*{Workshops}
\begin{enumerate}
\item Graphs and Surfaces: Algorithms, Combinatorics, and Topology. \\
  CIRM, Marseille, France, May 2016.
\item Symplectic and Algebraic Geometry in the Statistical Physics of Polymers. \\
  Simons Center for Geometry and Physics, Stony Brook NY, October 2015. 
\item PIMS-USASK Graduate Summer School in Applied Combinatorics. \\
  University of Saskatchewan, Saskatoon SK, May 2015.
\end{enumerate}

\bigskip

% Footer
\begin{center}
  \begin{footnotesize}
    Last updated: \today \\
    \href{\footerlink}{\texttt{\footerlink}}
  \end{footnotesize}
\end{center}

\end{document}

%%% Local Variables:
%%% mode: latex
%%% TeX-master: t
%%% End:
