\documentclass[12pt]{article} \usepackage{amsmath}

\pagestyle{empty} %<== DELETE IF YOUR TEST HAS MORE THAN ONE PAGE
% (this page style suppresses page numbers.)
\usepackage{hyperref}

% Adjust the margins:
\addtolength{\textwidth}{1.75in} \addtolength{\textheight}{1.75in}
\addtolength{\topmargin}{-.95in} \addtolength{\hoffset}{-.95in}
\setlength{\parskip}{1em}


% puts in a horizontal rule the width of the text with space before
% and after.
\newcommand{\sep}{\ifhmode\par\fi\vskip6pt\hrule\vskip6pt}

% A new environment for a test problem.
\newcounter{problem} \newenvironment{problem} { \stepcounter{problem}%
  \filbreak%
  \sep
  \noindent\arabic{problem}.
} { \filbreak }



\begin{document}
\begin{center}
  \textbf{WEBWORK--GETTING STARTED}
\end{center}
For this class, we are using a web-based homework system called WebWork. The login link is
\begin{center}
  \url{https://webwork.math.uga.edu/webwork2/Math2250\_Chapman\_S16}
\end{center}
You may have to tell your browser to accept a certificate in order to
access this page. Your username is your MyID from your official UGA
email address (so if you are bigdawg2011@uga.edu, then your username
is bigdawg2011). Your initial password is a variant of your 810
number, WITHOUT DASHES. So it
should look like 810123456. Note that this drops the last digit that
you see on your Student ID cards. You should change your password
after you login for the first time.

WebWork may be different from your previous experiences with math
homework. Until the assignment is due, you can try the problems as
many times as you like, and the system will tell you whether or not
you have the right answer. This lets you correct your work
immediately.

After the assignment's due date, the system will show you the correct
answer for each problem when you try it (but your answers will not be
scored). The funny thing about WebWork is that due dates are absolute;
i.e., \textbf{THERE ARE NO EXTENSIONS ON HOMEWORK.}

You are welcome to work together on WebWork problems, but be warned:
the problems are a little different for each student, so copying other
folks' answers won't work. When you first login to WebWork, you'll see
three buttons on the left. Use the ``Change Email" button to enter
your preferred email address and the ``Change Password" button to
change your password. Then, try ``Begin Problem Sets" to see how the
system works. You can select a set and print it out in PDF format to
work out the problems on paper if you prefer. Your problems will be
the same when you login again to enter the answers. Additionally,
there is a WeBWorK Intro under problem sets; this is a tutorial on how
WebWork works, and it is encouraged you complete it.

When typing things into WebWork, note that the system will use the
order of operations. Exponentiation comes first, multiplication and
division second, and addition and subtraction last. You can overrule
this sequence with parentheses. Expressions in parentheses are
evaluated first. Redundant parentheses are legal and possible. For
example, if your expression is $$\dfrac{a+b}{c-d}$$ you need to enter
into WebWork $$(a+b)/(c-d)$$

To this end, the ``Preview Answers" button is useful to make sure you
have typed your answer in correctly. When you click ``Submit Answer"
WebWork will tell you whether or not the answer you typed is correct.


\end{document}
