% LaTeX Curriculum Vitae Template
%
% Copyright (C) 2004-2009 Jason Blevins <jrblevin@sdf.lonestar.org>
% http://jblevins.org/projects/cv-template/
%
% You may use use this document as a template to create your own CV
% and you may redistribute the source code freely. No attribution is
% required in any resulting documents. I do ask that you please leave
% this notice and the above URL in the source code if you choose to
% redistribute this file.

\documentclass[letterpaper]{article}

\usepackage{hyperref}
\usepackage{geometry}
\usepackage{amsmath}
\usepackage{amsfonts}

% Comment the following lines to use the default Computer Modern font
% instead of the Palatino font provided by the mathpazo package.
% Remove the 'osf' bit if you don't like the old style figures.
%\usepackage[T1]{fontenc}
%\usepackage[sc,osf]{mathpazo}

% Set your name here
\def\name{Harrison Chapman}

% Replace this with a link to your CV if you like, or set it empty
% (as in \def\footerlink{}) to remove the link in the footer:
\def\footerlink{http://hchapman.github.io/static/Pubs.pdf}

% The following metadata will show up in the PDF properties
\hypersetup{
  colorlinks = true,
  urlcolor = black,
  pdfauthor = {\name},
  pdfkeywords = {mathematics, knot theory, random topology, statistical mechanics, physics},
  pdftitle = {\name: Publication List},
  pdfsubject = {Publication List},
  pdfpagemode = UseNone
}

\geometry{
  body={6.5in, 8.5in},
  left=1.0in,
  top=1.25in
}

% Customize page headers
\pagestyle{myheadings}
\markright{\name}
\thispagestyle{empty}

% Custom section fonts
\usepackage{sectsty}
\sectionfont{\rmfamily\mdseries\Large}
\subsectionfont{\rmfamily\mdseries\itshape\large}

% Other possible font commands include:
% \ttfamily for teletype,
% \sffamily for sans serif,
% \bfseries for bold,
% \scshape for small caps,
% \normalsize, \large, \Large, \LARGE sizes.

% Don't indent paragraphs.
\setlength\parindent{0em}

% Make lists without bullets
\renewenvironment{itemize}{
  \begin{list}{}{
    \setlength{\leftmargin}{1.5em}
  }
}{
  \end{list}
}

\begin{document}

% Place name at left
{\huge \name}\\
{\large\sc Publication List}

% Alternatively, print name centered and bold:
%\centerline{\huge \bf \name}

\begin{enumerate}
\item Markov chain sampling of random knot diagrams. In preparation. \\
  with A.\ Rechnitzer. \\
  \textit{We demonstrate a Markov Chain Monte Carlo sampler on the space of knot
    shadows and prove that it is ergodic. This provides a new way to uniformly
    sample knot diagrams and computationally enumerate the class of knot
    diagrams.}
\item Slipknotting in knot diagrams. In preparation. \\
  \textit{I apply pattern theorem results to knot diagrams to show that
    slipknots of any topological type appear almost surely in different classes
    of knot diagrams. Namely, I show that diagrams of unknots contain slipknots,
    a result which remains a conjecture for other models of knotting.}
\item Asymptotic laws for random knot diagrams. \emph{Submitted} 2016. \\
  Preprint: \href{http://arxiv.org/abs/1608.02638}{\tt arXiv:1608.02638}. \\
  \textit{I show a pattern theorem result for knot diagrams and use it to both
    prove the Frisch-Wasserman-Delbr\"uck conjecture for the model of knot
    diagrams and show that symmetries of knot diagrams are statistically
    insignificant.}
\item Knot probabilities in random diagrams.\\
  with J.\ Cantarella, M.\ Mastin. \\
  J. Phys. A: Math. Theor. 49 (2016), no. 40, p. 405001.\\
  DOI: \href{http://dx.doi.org/10.1088/1751-8113/49/40/405001}{\tt 10.1088/1751-8113/49/40/405001}.\\
  Preprint: \href{http://arxiv.org/abs/1512.05749}{\tt arXiv:1512.05749}. \\
  \textit{We compute the exact probability that a random diagram of
    $n$ crossings has a given knot type (for $n \leq 10$) by directly
    enumerating and classifying the knot diagrams of $n$
    crossings. Our enumeration method is based on identifying knot
    diagrams with a class of embedded 4-regular planar graphs.}
\item A Group-theoretic Approach to Human Solving Strategies in
  Sudoku.\\
  with M.\ Rupert (Mentor: E.\ Arnold). \\
  Colonial Academic Alliance Undergraduate Research Journal (2012) vol
  3, article 3.\\
  \textit{Paper produced during an NSF REU at James Madison University
    in 2010. We quantify the data of Sudoku board states by
    considering which numbers \emph{cannot} go in a given cell and
    consider how a typical player's solving strategies are a group
    acting on this set of states.}
\item On orbital varieties of type A. \\
  Advisor: T.\ Pietraho. Honors thesis (2011). Bowdoin
  College. \\
  \textit{Thesis on the Smith conjecture on orbital varieties of
    nilpotent orbits in the Lie group $GL_n$.  Outlines the
    correspondence between Young tableaux and orbital varieties and
    concludes with conditions for which certain shapes of Young
    tableaux will admit orbital varities which are not Smith.}
\end{enumerate}

\bigskip

% Footer
\begin{center}
  \begin{footnotesize}
    Last updated: \today \\
    \href{\footerlink}{\texttt{\footerlink}}
  \end{footnotesize}
\end{center}

\end{document}

%%% Local Variables:
%%% mode: latex
%%% TeX-master: t
%%% End:
